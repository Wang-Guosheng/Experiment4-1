\newtitle{连续与脉冲核磁共振}{王国胜}{201511140243}{廖红波}{2018}{3}{22}
\begin{abstract}
	本实验将分别用连续和脉冲核磁共振仪测量不同浓度CuSO\textsubscript{4}水溶液中$^1_1$H核的核磁共振表观横向弛豫时间$T_2^*$和横向弛豫时间$T_2$,由此确定它们随溶液浓度的变化关系,并测量甘油和二甲苯的相对化学位移。连续和脉冲核磁共振测试的结果表明,随着硫酸铜溶液浓度的增大,表观弛豫时间基本不变,而实际横向弛豫时间则近似按指数规律衰减,而这是由磁化强度进动和弛豫过程的性质决定的。连续法实验中还测量了磁场的不均匀度$\frac{\Delta B^*}{B_0}=4.79\times 10^{-5}$。最后将用脉冲核磁共振试验仪分析二甲苯和甘油的相对化学位移,得出二甲苯两个峰之间相对化学位移$\delta =1.171\times 10^{4}\,\mathrm{ppm}$,而甘油则观察不到化学位移;由此我们将进一步分析其与分子构型的关系。
\end{abstract}
\begin{newkeywords}{cccccc}
	核磁共振&拉莫尔进动&磁化强度弛豫过程&布洛赫方程&连续核磁共振波谱仪\\
	脉冲核磁共振波谱仪&自旋回波&$90^{\circ}-\tau-180^{\circ}$脉冲序列&化学位移&
\end{newkeywords}
