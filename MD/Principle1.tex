\section{实验原理}\label{ux5b9eux9a8cux539fux7406}

\subsection{量子力学表述}\label{ux91cfux5b50ux529bux5b66ux8868ux8ff0}

\subsubsection{原子核自旋}\label{ux539fux5b50ux6838ux81eaux65cb}

原子核自旋磁矩是其中核子磁矩的总和。
质子自旋为\(\dfrac{1}{2}\),自旋磁矩与自旋角动量同向;
而中子自旋为\(-\dfrac{1}{2}\),自旋磁矩与自旋角动量反向。由此可知\textbf{核自旋量子数}\(I\)符合以下规律:偶偶核\(I=0\),奇奇核\(I\)为整数,其它\(I\)为半整数。核自旋量子数为\(I\)的原子核\textbf{核自旋角动量}

\[
    \boldsymbol{P}_{I} = \sqrt{(I(I+1))}\hbar,
\]

故其\textbf{核自旋磁矩}为

\[
    \boldsymbol{\mu}_{I} = g_N\dfrac{e}{2m_{\text{p}}}\boldsymbol{P}_{I} = g_N\dfrac{e\hbar}{2m_{\text{p}}}\hat{\boldsymbol{\mu}}\_I\sqrt{(I(I+1))} = g_N\boldsymbol{\mu_N}\sqrt{(I(I+1))} = \gamma \boldsymbol{P}_{I}.
\]

式中\(g_N\)是核的朗德因子,\(m_{\text{p}}\)为核子质量,\(\boldsymbol{\mu_N}\)是玻尔磁子,\(\gamma\)是旋磁比。

\subsubsection{塞曼分裂}\label{ux585eux66fcux5206ux88c2}

在静磁场中,\(\boldsymbol{P}_I\)的取向量子化,\(\boldsymbol{\mu}_I\)也随之量子化,

\[
    P_z = m\_I\hbar\Rightarrow \mu_z = \gamma P_z=\gamma m_I\hbar,\quad m_I = I,I-1,\cdots,-I.
\]

由此产生的附加能量和相邻磁能级能量差为

\[
    E = -m_I\gamma\hbar B,\quad \Delta E = \gamma\hbar B = \hbar \omega. (\omega = \gamma B)
\]

\subsection{宏观理论}\label{ux5b8fux89c2ux7406ux8bba}

\subsubsection{单核拉莫尔进动}\label{ux5355ux6838ux62c9ux83abux5c14ux8fdbux52a8}

在\(P_z\)量子化的假定下可以得到经典模型

\[
    \dfrac{{\rm d}\boldsymbol{\mu}}{{\rm d}t} = \gamma\dfrac{{\rm d}\boldsymbol{P}}{{\rm d}t} = \gamma (\boldsymbol{\mu}\times \boldsymbol{B})
\]

this
