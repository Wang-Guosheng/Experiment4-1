\par 实验分别用连续和脉冲核磁共振仪测量了不同浓度CuSO\textsubscript{4}水溶液中$^1_1$H核的核磁共振表观横向弛豫时间$T_2^*$和横向弛豫时间$T_2$。连续核磁共振中,试样在扫描磁场达到共振磁场时发生核磁共振,共振信号的最大值随着溶液浓度增大而增大,较浓的硫酸铜溶液在共振磁场处之后表现出振幅指数衰减的振荡的吸收信号,较稀的溶液和纯水还表现出共振峰之前的色散信号;脉冲核磁共振实验中则可以观察到由$90^{\circ}-\tau -180^{\circ}$脉冲序列后出现自旋回波,说明不均匀磁场下进动速率不同的磁化强度分量可以重新汇合并继续表现弛豫过程,由此测量出不同浓度样品的表观和实际横向弛豫时间如\cref{tab:Pulse}。连续法测得的磁场不均匀度平均为$\dfrac{\Delta B^*}{B_0}=4.79\times 10^{-5}$。两种方法的实验结果都表明:随着硫酸铜溶液浓度的增大,表观弛豫时间基本不变,而实际横向弛豫时间则近似按指数规律衰减。这是由于溶液中的顺磁离子产生的局域磁场影响了氢核周围的磁场环境,加快了氢核的弛豫过程,而表观横向弛豫时间由于外磁场不均匀性的影响占主导,在此实验的精度下难以反映弛豫过程的影响。实验用脉冲核磁共振试验仪对二甲苯和甘油的相对化学位移测量,得出二甲苯两个峰之间相对化学位移$\delta =1.171\times 10^{4}\,\mathrm{ppm}$,而甘油则观察不到化学位移,这说明前者的分子构型包含两种氢原子,后者则只包含一种独立的氢原子。