\subsection{仪器连接准备}\label{ux4eeaux5668ux8fdeux63a5ux51c6ux5907}

先预热:打开脉冲核磁共振温度控制主机后面板的电源开关,过一段时间可以看到温度升高,经过
3-4 个小时,磁铁温度稳定在 36.50 摄氏度(有时会在 36.44 摄氏度 36.56
摄氏度之间变化,属正常现象)。将 5%CuSO\textsubscript{4}
水溶液样品放入探头内,插入磁铁间隙中。将磁场扫描电源上的扫描输出的两个输出端与磁铁面板中的扫描电源接线柱连接,将共振仪后面板上的相移输出连接到示波器的
CHI, 而放大器电源与边限振荡器四芯电源接头连接;将边限振荡器的信号输出用
Q9 线接示波器的 CH2, 频率计输出用 Q9 线接频率计的 A
通道(频率计的通道选择:A 通道,即 1Hz-100MHz;FUNCTION 选择:FA;GATE
TIME 选择:1s。打开磁场扫描电源、频率计和示波器的电源。

\subsection{核磁共振信号的调节}\label{ux6838ux78c1ux5171ux632fux4fe1ux53f7ux7684ux8c03ux8282}

\begin{itemize}
\item
  调节边限振荡器的频率粗调电位器,待发现信号之后,旋动频率调节细调旋钮,在此附近捕捉信号,当满足共振条件$\omega_0=\gamma B_0$
时,可以观察到。调节旋钮时要尽量慢,因为共振范围非常小,很容易跳过。
\item
  调出大致共振信号后,同时调节样品在磁隙中的位置以得到尾波最多的共振信号,此时样品处于磁场较均匀的位置,因为尾波越多对应$T_2^{*}$越长,即$\Delta B^{*}$  越小。调节频率微调至信号等宽,此时频率计上的频率就是样品的共振频率,因为信号等宽说明同一周期中的两次共振都发生在扫场波形的节点处,对应$B_z = B_0$。记录下此时样品信号的形状和信号的最大值。
\item
  根据{\color{red} ref}估算样品的表观横向弛豫时间。
\item
  磁场均匀度测量:根据{\color{red} ref},测量相关的实验参数,并计算磁场的非均匀度,扫描频率为50Hz。
\item
  按同样的方法分别测量 1%、0.5%、0.05%的 CuSO\textsubscript{4}水溶液和纯水的共振信号的形状、信号最大值、$T_2$相互比较。
\item
  实验中应保持样品位置不变,测量前检查样品是否充满样品池。
\item
  氢的旋磁比为 42.577MHz。
\end{itemize}

\subsection{脉冲核磁共振}\label{ux8109ux51b2ux6838ux78c1ux5171ux632f}

\begin{itemize}
\item
  仪器连接准备:检查各实验仪器是否连接正确。
\item
  信号粗调:将 0.05\%CuSO\textsubscript{4} 样品放入样品池,关好盖板。
\item
  打开 PNMR 脉冲核磁采集软件,点击连续采集按钮,电脑控制发出射频信号,频率一般在
  20mHz,另外初始值一般为脉冲间隔 0.20ms,第一脉冲宽度
  10-20ms,第二脉冲宽度 30∼
  60ms。在软件给出的初始条件下,仔细调节磁铁调场电源,小范围改变磁场,当调至合适值时,可以在采集软件界面中观察到共振信号(调节合适也可以观察到自旋回波信号)。这时调节主机面板上磁铁匀场电源使磁场均匀度最佳,即信号尾波最多、自旋回波最强。
\item
  用自旋回波法测量横向弛豫时间:在上一步的基础上,将脉冲间隔调节至调节第一脉冲宽度使其作用时间满足
  90$^{\circ}$脉冲条件(判断依据:第一 FID
  信号最大),并调节第二脉冲宽度至第一脉冲宽度的两倍左右(因为仪器本身特性,并不完全是两倍关系)作为
  180$^{\circ}$
  脉冲,然后仔细并反复调节匀场电源、调场电源、第一脉冲宽度和第二脉冲宽度。使自旋回波信号最大。在上一步的基础上,将脉冲间隔调节至调节第一脉冲宽度使其作用时间满足
  90$^{\circ}$脉冲条件(判断依据:第一 FID
  信号最大),并调节第二脉冲宽度至第一脉冲宽度的两倍左右(因为仪器本身特性,并不完全是两倍关系)作为
  180$^{\circ}$
  脉冲,然后仔细并反复调节匀场电源、调场电源、第一脉冲宽度和第二脉冲宽度。使自旋回波信号最大。应用软件测量不同脉冲间隔情况下(至少取
  6 个间隔)的回波信号大小,进行指数拟合得到横向弛豫时间。
\item
  自由感应衰减 (FID) 信号测量表观横向弛豫时间。
\end{itemize}

将脉冲间隔调节至最大(60ms),第二脉冲宽度调节至
0ms,只剩下第一脉冲,仔细调节调场电源和匀场电源(电源粗调和电源细调结合起来用),并小范围调节第一脉冲宽度,使尾波最大,应用软件通过指数拟合测量表观横向弛豫时间。

\begin{itemize}
\item
  换取不同浓度 CuSO\textsubscript{4}
  水溶液样品(0.1%,0.3%,0.5%,1%,5%),分别测量样品的横向弛豫时间和表观横向弛豫时间。
\item
  测量样品的相对化学位移:分别放入甘油和二甲苯样品,调节匀场电源、调场电源使自旋回波信号最大,然后分别测量两种样品的相对化学位移。
\end{itemize}
