\par 布洛赫(Felix Bloch)和珀塞尔(Edward Purcell)分别在1945年发明了核磁共振技术,大大提高了核磁矩的测量精度。核磁共振技术迅速、准确、分辨率高,不破坏样品,已经在许多研究和生产领域广泛应用。\supct{bib:textbook}两人最初发明此技术时采用的都是连续信号;现代则一般采用脉冲信号组成脉冲序列,因为后者可以避免外加恒定磁场不均匀性的影响,并在短时间内集中更高的能量,增大信号强度。\supct{bib:mit}
本实验中,我们将以水、硫酸铜水溶液和甘油与二甲苯样品中的氢核为例,用连续和脉冲两种核磁共振方法分别观测核磁共振现象和各种因素对共振信号的影响,并从量子模型和准经典模型两个角度说明其成因。