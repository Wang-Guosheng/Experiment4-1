%连续核磁共振
	%表观横向弛豫时间与样品浓度的关系。
\subsection{连续核磁共振实验} % (fold)
	\label{sub:连续核磁共振实验}
	
% subsection 连续核磁共振实验 (end)
%脉冲核磁共振实验
	%横向弛豫时间与样品浓度的关系。
%结合连续核磁共振和脉冲核磁共振的实验结果,以及讲义中的公式44,解释表观横向弛豫时间和横向弛豫时间与CuSO\(_4\)水溶液浓度变化关系差异较大的原因。
%结合在脉冲核磁实验中获得的横向弛豫时间与浓度的关系,分析在连续核磁共振实验中,不同浓度CuSO\(_4\)水溶液样品共振信号形状不同的原因。提示:比较扫场周期和横向弛豫时间不的大小关系和
%分析为什么甘油的核磁共振信号没有化学位移而二甲苯却有?提示:两者的分子结构不同。